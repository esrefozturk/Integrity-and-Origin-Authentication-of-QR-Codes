

\documentclass[12pt,journal,compsoc]{IEEEtran}

\begin{document}

\title{
	\huge TITLE \\     
    \Large Computer Engineering Department, METU}


\author{\normalsize Ozge~Lule(1881366),~Esref~Ozturk(1881762)}


\IEEEtitleabstractindextext{%
\begin{abstract}

During recent years QR codes started to be used in many areas including posters, business and personal cards and mobile application links. However it brings a danger with itself, that is no one can be sure if qr code is generated  by who. This report is intented  to make a solution to this problem with origin authentication extension to QR codes.

\end{abstract}


\begin{IEEEkeywords}

 KEYWORDS

\end{IEEEkeywords}}


% make the title area
\maketitle

\IEEEdisplaynontitleabstractindextext

\IEEEpeerreviewmaketitle

\section{Introduction}

\IEEEPARstart{Q}{R} code (Quick Response Code) is the trademark for a type of matrix barcode (or two-dimensional barcode) first designed for the automotive industry in Japan. A barcode is a machine-readable optical label that contains information about the item to which it is attached. A QR code uses four standardized encoding modes (numeric, alphanumeric, byte/binary, and kanji) to efficiently store data; extensions may also be used.

A QR code consists of black modules (square dots) arranged in a square grid on a white background, which can be read by an imaging device (such as a camera, scanner, etc.) and processed using Reed–Solomon error correction until the image can be appropriately interpreted. The required data are then extracted from patterns that are present in both horizontal and vertical components of the image.





\begin{thebibliography}{1}

REFERENCES

\end{thebibliography}





\end{document}

